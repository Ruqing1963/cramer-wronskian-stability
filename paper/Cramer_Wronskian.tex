% ═══════════════════════════════════════════════════════════════════════
% Radical Compression in Maximal Prime Gaps:
% Asymptotics and the Transient Stability Illusion
% Titan Project — Paper XI (revised) — February 2026
% ═══════════════════════════════════════════════════════════════════════

\documentclass[11pt, a4paper]{article}

\usepackage[top=28mm, bottom=28mm, left=25mm, right=25mm]{geometry}
\usepackage[T1]{fontenc}
\usepackage{amsmath, amssymb, amsthm, mathtools}
\usepackage{mathrsfs}
\usepackage[dvipsnames]{xcolor}
\usepackage{enumitem}
\usepackage{booktabs}
\usepackage{hyperref}
\usepackage{float}

\newtheorem{theorem}{Theorem}[section]
\newtheorem{lemma}[theorem]{Lemma}
\newtheorem{proposition}[theorem]{Proposition}
\newtheorem{corollary}[theorem]{Corollary}
\theoremstyle{definition}
\newtheorem{definition}[theorem]{Definition}
\newtheorem{observation}[theorem]{Observation}
\newtheorem{question}[theorem]{Open Question}
\theoremstyle{remark}
\newtheorem{remark}[theorem]{Remark}

\newcommand{\rad}{\mathrm{rad}}

\title{\textbf{Radical Compression in Maximal Prime Gaps: \\
Asymptotics and the Transient Stability Illusion}}
\author{Ruqing Chen\\[4pt]
\small GUT Geoservice Inc., Montr\'eal, Canada\\[2pt]
\small \texttt{ruqing@hotmail.com}\\[2pt]
\small Repository:
\url{https://github.com/Ruqing1963/cramer-wronskian-stability}}
\date{February 2026}

\begin{document}

\maketitle

\begin{abstract}
For a prime gap of length $g$ starting at a prime $p$,
we define the \emph{radical compression ratio}
$q = \log(\prod C_i) / \log \rad(\prod C_i)$,
where the product ranges over the $g - 1$ consecutive
composites in the gap.
This ratio measures the extent to which composites in a
prime gap share prime factors.
We prove, using Legendre's formula and Mertens' estimates,
that
\begin{equation*}
    q \;=\; 1 + \frac{\log g}{\log X} + O\!\left(\frac{1}{\log X}\right),
\end{equation*}
where $X$ is the starting prime.
In the Cram\'er regime $g \sim c\log^2 X$, this yields
$q = 1 + \frac{2\log\log X}{\log X} + O(1/\log X) \to 1$
as $X \to \infty$.
Thus, no finite ``stabilization'' of $q$ occurs;
the apparent stability near $q \approx 1.30$ observed for
$X \le 10^7$ is a transient artifact of the extremely slow
decay of $\log\log X / \log X$.
We verify this asymptotic formula against all 25
maximal prime gaps up to $10^8$, finding agreement to
within $3\%$ for $g \ge 36$.
The radical compression ratio provides a clean
multiplicative perspective on prime gaps that complements
the classical additive/probabilistic approach of Cram\'er.
\end{abstract}

\bigskip

% ═══════════════════════════════════════════════════════════════════════
\section{Introduction}\label{sec:intro}
% ═══════════════════════════════════════════════════════════════════════

\subsection{Motivation}

Cram\'er's conjecture \cite{Cramer1936} asserts that
$g_n = O((\log p_n)^2)$,
where $g_n = p_{n+1} - p_n$ is the $n$-th prime gap.
The conjecture is supported by a probabilistic model in
which each integer $m$ is ``prime'' independently with
probability $1/\log m$.
This model is purely additive: it treats primality as a
coin toss and ignores the multiplicative anatomy of the
composites that fill a prime gap.

In this paper, we take the complementary multiplicative
viewpoint.
We ask: \emph{how much do the composites in a prime gap
share their prime factors?}
We quantify this via the \emph{radical compression ratio}
$q$, which compares the log-product of the composites to
the log-radical of their collective product.

Our main result is that $q$ admits an explicit asymptotic
formula in terms of $g$ and $X$ (Theorem~\ref{thm:main}),
from which $q \to 1$ as $X \to \infty$ in the
Cram\'er regime.
This resolves the initial impression from limited numerical
data that $q$ ``stabilizes'' around $1.3$---an illusion
created by the notoriously slow decay of the iterated
logarithm $\log\log X / \log X$.

\subsection{Context within the Titan Project}

In companion papers, we proved algebraic structure theorems
for Legendre and Oppermann intervals:
the Spin Asymmetry Theorem \cite{ChenSpin}
(global quadratic residue skewness $= 1$)
and the Oppermann Parity Law \cite{ChenOpp}
(zero right-half skewness for even $n$).
Those results concern the additive/algebraic axis
(residue classes modulo $p = 2n - 1$).
The present paper addresses the multiplicative axis
(radical of consecutive composites), completing a
two-sided view of the arithmetic structure of prime gaps.

% ═══════════════════════════════════════════════════════════════════════
\section{Definitions}\label{sec:definitions}
% ═══════════════════════════════════════════════════════════════════════

\begin{definition}[Radical compression ratio]\label{def:q}
For a prime gap of length $g$ starting at a prime $p$,
the \emph{composite sequence} is
$\mathcal{C} = \{p+1, p+2, \ldots, p+g-1\}$.
The \emph{radical compression ratio} is
\begin{equation}\label{eq:q}
    q(p, g)
    \;=\;
    \frac{N}{D}
    \;=\;
    \frac{\displaystyle\sum_{c \in \mathcal{C}} \log c}
    {\displaystyle\sum_{\ell \in \mathcal{P}(\mathcal{C})} \log \ell},
\end{equation}
where $\mathcal{P}(\mathcal{C})
= \{\ell \text{ prime} : \ell \mid c \text{ for some }
c \in \mathcal{C}\}$
is the set of all distinct prime factors of the sequence.
\end{definition}

\begin{remark}[Relation to the ABC quality]\label{rem:abc}
For a single ABC triple $(a,b,c)$ with $a + b = c$,
the \emph{quality} is
$q(a,b,c) = \log c / \log \rad(abc)$.
Our ratio $q(p,g)$ is analogous but applied to
a \emph{product} of a sequence.
It is not an ABC quality, and we make no claims
involving the ABC conjecture.
\end{remark}

% ═══════════════════════════════════════════════════════════════════════
\section{Asymptotic Analysis}\label{sec:asymptotics}
% ═══════════════════════════════════════════════════════════════════════

\begin{theorem}[Radical compression asymptotics]
\label{thm:main}
Let $\mathcal{C}$ be a sequence of $g - 1$ consecutive
integers near $X$ (all elements in $[X, X+g]$),
with $g \ge 2$ and $X \ge g^2$.
Then
\begin{equation}\label{eq:asymp}
    q \;=\; 1 + \frac{\log g}{\log X}
    + O\!\left(\frac{1}{\log X}\right).
\end{equation}
In the Cram\'er regime $g \sim c (\log X)^2$:
\begin{equation}\label{eq:cramer_regime}
    q \;=\; 1 + \frac{2\log\log X + \log c}{\log X}
    + O\!\left(\frac{1}{\log X}\right)
    \;\xrightarrow{X \to \infty}\; 1.
\end{equation}
\end{theorem}

\begin{proof}
We compute the numerator $N$ and the excess $N - D$.

\textbf{Numerator.}
$N = \sum_{c \in \mathcal{C}} \log c
= (g-1)\log X + O(g)$,
since $\log(X + j) = \log X + O(g/X)$ for $0 \le j \le g$.

\textbf{Excess via Legendre's formula.}
The excess $N - D$ equals
$\sum_{\ell \mid \prod \mathcal{C}}
(v_\ell(\prod \mathcal{C}) - 1) \log \ell$,
where $v_\ell$ denotes the $\ell$-adic valuation.

\emph{Small primes} ($\ell \le g$):
By Legendre's formula,
$v_\ell(\prod \mathcal{C})
= \sum_{k \ge 1} \lfloor(g-1)/\ell^k\rfloor
= (g-1)/(\ell - 1) + O(\log_\ell g)$.
Thus the small-prime contribution to $N - D$ is
\begin{align*}
    \sum_{\ell \le g}
    \left(\frac{g-1}{\ell - 1} - 1\right) \log \ell
    &= (g-1)\sum_{\ell \le g} \frac{\log \ell}{\ell - 1}
       - \theta(g) + O(g) \\
    &= (g-1)(\log g + O(1)) - (g + o(g)) + O(g) \\
    &= g\log g + O(g),
\end{align*}
using Mertens' estimate
$\sum_{\ell \le x} \log \ell / (\ell - 1) = \log x + O(1)$
and Chebyshev's $\theta(g) = g + o(g)$.

\emph{Large primes} ($\ell > g$):
Each such prime divides at most one element of $\mathcal{C}$,
so the excess is nonzero only from prime powers
$\ell^k \mid c$ with $k \ge 2$.
This contributes $O(g)$ in total.

\textbf{Combining.}
$N - D = g\log g + O(g)$, so
$D = g\log X - g\log g + O(g) = g(\log X - \log g) + O(g)$.
Therefore
\[
    q = \frac{N}{D}
    = \frac{g\log X + O(g)}{g(\log X - \log g) + O(g)}
    = \frac{\log X}{\log X - \log g + O(1)}
    = 1 + \frac{\log g}{\log X}
      + O\!\left(\frac{1}{\log X}\right).
\]
Substituting $g = c\log^2 X$ gives \eqref{eq:cramer_regime}.
\end{proof}

\begin{corollary}\label{cor:limit}
If the Cram\'er conjecture holds, then the radical
compression of maximal prime gaps satisfies
$\lim_{n\to\infty} q = 1$.
\end{corollary}

\begin{remark}[Rate of decay]\label{rem:rate}
The decay $2\log\log X / \log X$ is notoriously slow:
\begin{center}
\renewcommand{\arraystretch}{1.1}
\begin{tabular}{@{}l r@{}}
\toprule
$X$ & $1 + 2\log\log X / \log X$ \\
\midrule
$10^{6}$    & 1.380 \\
$10^{12}$   & 1.240 \\
$10^{50}$   & 1.082 \\
$10^{100}$  & 1.047 \\
\bottomrule
\end{tabular}
\end{center}
The passage from $q \approx 1.30$ to $q \approx 1.05$
requires $X$ to increase from $10^7$ to $10^{100}$.
This explains why limited numerical experiments create
an illusion of ``stabilization.''
\end{remark}

% ═══════════════════════════════════════════════════════════════════════
\section{Computational Verification}\label{sec:data}
% ═══════════════════════════════════════════════════════════════════════

\subsection{Methodology}

We enumerated all primes up to $10^8$ via sieve,
identified all 25 maximal (record-breaking) prime gaps,
and computed $q$ by fully factoring every composite.

\subsection{Data}

\begin{table}[H]
\centering
\caption{Maximal prime gaps with $g \ge 20$ up to $10^8$.
$q_{\mathrm{obs}}$ = observed;
$q_{\mathrm{pred}} = 1 + \log g / \log P_n$.}
\label{tab:data}
\medskip
\begin{tabular}{@{}r r r r r r@{}}
\toprule
$P_n$ & $g$ & $\rho$ & $q_{\mathrm{obs}}$ &
$q_{\mathrm{pred}}$ & Error \\
\midrule
887         & 20  & 0.434 & 1.421 & 1.441 & $-1.4\%$ \\
1\,129      & 22  & 0.445 & 1.376 & 1.440 & $-4.5\%$ \\
1\,327      & 34  & 0.658 & 1.541 & 1.490 & $+3.4\%$ \\
9\,551      & 36  & 0.429 & 1.371 & 1.391 & $-1.5\%$ \\
15\,683     & 44  & 0.472 & 1.382 & 1.392 & $-0.7\%$ \\
19\,609     & 52  & 0.532 & 1.374 & 1.400 & $-1.8\%$ \\
31\,397     & 72  & 0.672 & 1.424 & 1.413 & $+0.8\%$ \\
155\,921    & 86  & 0.602 & 1.377 & 1.373 & $+0.3\%$ \\
360\,653    & 96  & 0.586 & 1.346 & 1.357 & $-0.8\%$ \\
370\,261    & 112 & 0.681 & 1.371 & 1.368 & $+0.2\%$ \\
492\,113    & 114 & 0.664 & 1.343 & 1.361 & $-1.4\%$ \\
1\,349\,533 & 118 & 0.592 & 1.333 & 1.338 & $-0.4\%$ \\
1\,357\,201 & 132 & 0.662 & 1.336 & 1.346 & $-0.7\%$ \\
2\,010\,733 & 148 & 0.703 & 1.339 & 1.344 & $-0.4\%$ \\
4\,652\,353 & 154 & 0.653 & 1.304 & 1.328 & $-1.8\%$ \\
17\,051\,707& 180 & 0.649 & 1.294 & 1.312 & $-1.3\%$ \\
20\,831\,323& 210 & 0.740 & 1.306 & 1.317 & $-0.8\%$ \\
47\,326\,693& 220 & 0.704 & 1.294 & 1.305 & $-0.8\%$ \\
\bottomrule
\end{tabular}
\end{table}

\begin{observation}[Agreement with asymptotics]
\label{obs:agreement}
For all $g \ge 36$ (15 data points), the prediction
$q_{\mathrm{pred}} = 1 + \log g / \log X$
agrees with $q_{\mathrm{obs}}$ to within $3.5\%$.
The systematic negative bias is consistent with the
$O(1/\log X)$ correction term.
\end{observation}

\begin{observation}[Steady decline]\label{obs:decline}
$q_{\mathrm{obs}}$ decreases from $1.54$ at
$P_n = 1327$ to $1.29$ at $P_n = 47{,}326{,}693$,
consistent with the proven asymptotic decay to $1$.
The apparent plateau near $1.30$ for
$X \in [10^6, 10^7]$ is a transient artifact.
\end{observation}

% ═══════════════════════════════════════════════════════════════════════
\section{Discussion}\label{sec:discussion}
% ═══════════════════════════════════════════════════════════════════════

\subsection{The decompression mechanism}

The formula $q = 1 + \log g/\log X + O(1/\log X)$
has a clean interpretation.
The excess $N - D \approx g\log g$ is generated by
\emph{small} primes ($\ell \le g$), each dividing multiple
composites in the gap.
But the denominator $D$ is dominated by \emph{large}
prime factors ($\ell > g$), each unique to a single
composite and contributing $\sim \tfrac{1}{2}\log X$
to $D$.
As $X$ grows, these large primes increasingly dilute
the small-prime compression, driving $q \to 1$.

\subsection{Relation to the ABC conjecture}

The ABC conjecture bounds the quality of individual triples
$a + b = c$.
The radical compression $q$ measures the collective
compression of a \emph{sequence}.
The boundedness of $q$ follows from elementary analysis
(Theorem~\ref{thm:main}), not from the ABC conjecture.

\subsection{Open questions}

\begin{question}[Explicit error term]\label{q:explicit}
Can the $O(1/\log X)$ in \eqref{eq:asymp} be made
explicit, e.g., $q = 1 + \log g/\log X + C/\log X + o(1/\log X)$
for an identifiable constant $C$?
\end{question}

\begin{question}[Oscillation structure]\label{q:oscillation}
The residual $q_{\mathrm{obs}} - q_{\mathrm{pred}}$
oscillates between $-4.5\%$ and $+3.4\%$.
Is there a secondary term capturing these fluctuations?
\end{question}

\begin{question}[Non-record gaps]\label{q:nonrecord}
The proof of Theorem~\ref{thm:main} applies to any
$g - 1$ consecutive integers, not just composites.
For typical (non-record) prime gaps, how does $q$
behave?
\end{question}

\subsection*{Data and code availability}

Scanning scripts and the complete dataset are at
\url{https://github.com/Ruqing1963/cramer-wronskian-stability}.

% ═══════════════════════════════════════════════════════════════════════
\begin{thebibliography}{99}

\bibitem{ChenSpin}
  R.~Chen,
  ``Algebraic rigidity and quadratic residue asymmetry
  in Legendre intervals,''
  Zenodo, 2026.
  \url{https://zenodo.org/records/18706876}

\bibitem{ChenOpp}
  R.~Chen,
  ``Oppermann's parity law: quadratic residue symmetry
  breaking in half-intervals via the negation involution,''
  Zenodo, 2026.
  \url{https://zenodo.org/records/18707265}

\bibitem{ChenFF}
  R.~Chen,
  ``The geometry of prime vacuums: Legendre's conjecture
  in function fields via monodromy and Chebotarev density,''
  Zenodo, 2026.
  \url{https://zenodo.org/records/18705744}

\bibitem{Cramer1936}
  H.~Cram\'er,
  ``On the order of magnitude of the difference between
  consecutive prime numbers,''
  \textit{Acta Arith.}\ \textbf{2} (1936), 23--46.

\bibitem{Granville1995}
  A.~Granville,
  ``Harald Cram\'er and the distribution of prime numbers,''
  \textit{Scand.\ Actuar.\ J.}\ \textbf{1995} (1995), 12--28.

\end{thebibliography}

\end{document}
