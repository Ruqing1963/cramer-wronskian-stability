% ═══════════════════════════════════════════════════════════════════════
% The Discrete Wronskian Stability of Prime Gaps:
% Empirical Observations near the Cramér Limit
% Titan Project — Paper XI — February 2026
% ═══════════════════════════════════════════════════════════════════════

\documentclass[11pt, a4paper]{article}

\usepackage[top=28mm, bottom=28mm, left=25mm, right=25mm]{geometry}
\usepackage[T1]{fontenc}
\usepackage{amsmath, amssymb, amsthm, mathtools}
\usepackage{mathrsfs}
\usepackage[dvipsnames]{xcolor}
\usepackage{enumitem}
\usepackage{booktabs}
\usepackage{hyperref}
\usepackage{float}

\newtheorem{theorem}{Theorem}[section]
\newtheorem{lemma}[theorem]{Lemma}
\newtheorem{proposition}[theorem]{Proposition}
\theoremstyle{definition}
\newtheorem{definition}[theorem]{Definition}
\newtheorem{example}[theorem]{Example}
\newtheorem{observation}[theorem]{Observation}
\newtheorem{question}[theorem]{Open Question}
\theoremstyle{remark}
\newtheorem{remark}[theorem]{Remark}

\newcommand{\rad}{\mathrm{rad}}

\title{\textbf{The Discrete Wronskian Stability of Prime Gaps: \\
Empirical Observations near the Cram\'er Limit}}
\author{Ruqing Chen\\[4pt]
\small GUT Geoservice Inc., Montr\'eal, Canada\\[2pt]
\small \texttt{ruqing@hotmail.com}\\[2pt]
\small Repository:
\url{https://github.com/Ruqing1963/cramer-wronskian-stability}}
\date{February 2026}

\begin{document}

\maketitle

\begin{abstract}
We introduce the \emph{discrete Wronskian compression ratio}
$q_W$ for a prime gap: given a maximal gap of length $g$
starting at a prime $p$, the $g - 1$ consecutive composites
$\{p+1, \ldots, p+g-1\}$ have a collective radical
$R = \rad\bigl(\prod C_i\bigr)$,
and we define $q_W = \sum \log C_i / \log R$.
This ratio measures the degree to which the composites in the
gap share prime factors.
We report a systematic computational survey of all 18 maximal
(record-breaking) prime gaps up to $10^7$, finding a striking
empirical phenomenon: while the gap lengths $g$ grow in
accordance with the Cram\'er heuristic
($g \approx 0.65\, \log^2 p$),
the Wronskian compression $q_W$ does \emph{not} grow.
Instead, $q_W$ decays from initial values near $1.6$
for small gaps and stabilizes into a narrow band
$q_W \in [1.30, \, 1.43]$ for all record gaps with $g \ge 36$.
We propose this stabilization as a new empirical phenomenon
worthy of theoretical explanation, and formulate it as an
open conjecture: the Wronskian compression of maximal prime
gaps is bounded.
No claims conditional on the ABC conjecture are made.
\end{abstract}

\bigskip

% ═══════════════════════════════════════════════════════════════════════
\section{Introduction}\label{sec:intro}
% ═══════════════════════════════════════════════════════════════════════

\subsection{Cramér's conjecture and the random model}

Let $p_n$ denote the $n$-th prime and $g_n = p_{n+1} - p_n$
the $n$-th prime gap.
Cram\'er's conjecture (1936) asserts that
$\limsup_{n \to \infty} g_n / (\log p_n)^2 = 1$,
motivated by a probabilistic model in which each integer $m$
is ``prime'' independently with probability $1/\log m$.
While this model successfully predicts many statistical
properties of primes, it says nothing about the
\emph{multiplicative structure} of the composites that
fill a prime gap.

\subsection{The Wronskian perspective}

We propose a complementary viewpoint.
Rather than asking ``how long can a prime gap be?''
(a question about \emph{additive} structure), we ask:
``how multiplicatively compressed can a long sequence of
consecutive composites be?''

A sequence of composites in a prime gap shares prime factors
in a structured way: every other number is even, every third
is divisible by $3$, and so on.
The \emph{discrete Wronskian compression ratio} $q_W$
quantifies this sharing.
If the composites were pairwise coprime and squarefree,
$q_W = 1$; the more they share factors (or contain prime
powers), the higher $q_W$ becomes.

Our central empirical finding is that $q_W$ does not grow
with gap length.
This is not obvious a priori: longer gaps might be expected
to produce more factor-sharing and hence higher compression.
Instead, $q_W$ stabilizes, suggesting a structural equilibrium
in the multiplicative fabric of the integers.

\subsection{What this paper does and does not claim}

\emph{This paper reports an empirical observation and poses
an open problem.}
We do not claim to prove Cram\'er's conjecture, nor do we
invoke the ABC conjecture.
The Wronskian stabilization is presented as a new
\emph{phenomenon} in need of theoretical explanation.

\subsection{Context within the Titan Project}

In \cite{ChenSpin} and \cite{ChenOpp}, we proved rigorous
algebraic theorems about the quadratic residue structure of
Legendre and Oppermann intervals.
Those results concern the \emph{additive/algebraic} structure
of short intervals near $n^2$.
The present paper shifts to the \emph{multiplicative} structure
of prime gaps, complementing the algebraic perspective with
an arithmetic one.
The function field analogue of Legendre's conjecture was
resolved in \cite{ChenFF}.

% ═══════════════════════════════════════════════════════════════════════
\section{Definitions}\label{sec:definitions}
% ═══════════════════════════════════════════════════════════════════════

\begin{definition}[Maximal prime gap]\label{def:gap}
A prime gap $g_n = p_{n+1} - p_n$ is \emph{maximal}
(or \emph{record-breaking}) if $g_n > g_k$ for all $k < n$.
\end{definition}

\begin{definition}[Discrete Wronskian compression]
\label{def:wronskian}
For a prime gap of length $g$ starting at a prime $p$,
the \emph{composite sequence} is
$\mathcal{C} = \{p+1, p+2, \ldots, p+g-1\}$
(all $g - 1$ composites).
The \emph{discrete Wronskian compression ratio} is
\begin{equation}\label{eq:qw}
    q_W(p, g)
    \;=\;
    \frac{\displaystyle\sum_{c \in \mathcal{C}} \log c}
    {\displaystyle\log \rad\!\left(\prod_{c \in \mathcal{C}} c\right)}
    \;=\;
    \frac{\log\!\left(\prod_{c \in \mathcal{C}} c\right)}
    {\displaystyle\sum_{q \in \mathcal{P}(\mathcal{C})} \log q},
\end{equation}
where $\mathcal{P}(\mathcal{C})
= \{q \text{ prime} : q \mid c \text{ for some } c \in \mathcal{C}\}$
is the set of all distinct prime factors appearing in the
composite sequence.
\end{definition}

\begin{remark}[Interpretation of $q_W$]\label{rem:interp}
$q_W = 1$ if and only if $\prod c$ is squarefree
(equivalently, every prime divides at most one $c \in \mathcal{C}$
and every $c$ is squarefree).
$q_W > 1$ measures the ``excess'' coming from two sources:
\begin{enumerate}[nosep, label=(\alph*)]
\item \emph{Factor sharing:} a prime $q$ divides multiple
elements of $\mathcal{C}$ (e.g., $q = 2$ divides roughly
half the composites).
\item \emph{Prime powers:} an element $c = q^k m$ contributes
$k \log q$ to the numerator but only $\log q$ to the
denominator.
\end{enumerate}
Factor sharing is the dominant contributor for long gaps.
\end{remark}

\begin{definition}[Cramér ratio]\label{def:cramer}
The \emph{Cram\'er ratio} of a prime gap is
$\rho(p, g) = g / (\log p)^2$.
Cram\'er's conjecture predicts $\limsup \rho = 1$.
\end{definition}

% ═══════════════════════════════════════════════════════════════════════
\section{Elementary Bounds on $q_W$}\label{sec:bounds}
% ═══════════════════════════════════════════════════════════════════════

Before presenting the data, we establish simple bounds
to calibrate expectations.

\begin{proposition}[Lower bound]\label{prop:lower}
For any composite sequence of length $g - 1$ near $X$:
\[
    q_W \;\ge\; 1.
\]
Equality holds if and only if $\prod c$ is squarefree.
\end{proposition}

\begin{proof}
$\log \rad(N) \le \log N$ for all $N \ge 1$,
with equality iff $N$ is squarefree.
\end{proof}

\begin{proposition}[Heuristic upper bound via Mertens]
\label{prop:upper}
For a gap of length $g$ near $X$, the numerator is
$\sum \log c \approx (g-1) \log X$.
The denominator satisfies
$\sum_{q \in \mathcal{P}} \log q
\ge \sum_{\substack{q \le g \\ q \text{ prime}}} \log q
\approx g$ (by Chebyshev's bound / PNT for the Chebyshev
function $\theta(g) \sim g$).
Hence
\[
    q_W \;\lesssim\; \frac{(g-1)\log X}{g}
    \;\approx\; \log X
\]
in the worst case.
However, the denominator also includes primes $q > g$
that divide elements of $\mathcal{C}$, which significantly
increases it in practice.
\end{proposition}

\begin{remark}
The crude upper bound $q_W \lesssim \log X$ is never
remotely approached in practice, because large primes
contribute substantially to the radical.
The empirical finding (Section~\ref{sec:data}) is that
$q_W$ stays in $[1.3, 1.7]$, far below $\log X \sim 15$--$17$.
\end{remark}

% ═══════════════════════════════════════════════════════════════════════
\section{Computational Survey}\label{sec:data}
% ═══════════════════════════════════════════════════════════════════════

\subsection{Methodology}

We enumerated all primes up to $10^7$ using a sieve,
identified all 18 maximal prime gaps in this range,
and computed $q_W$ for each by fully factoring every
composite in the gap sequence.

\subsection{Complete data}

\begin{table}[H]
\centering
\caption{All 18 maximal prime gaps up to $10^7$.
$P_n$ = starting prime; $g$ = gap length;
$\rho$ = Cram\'er ratio $g/\log^2 P_n$;
$q_W$ = Wronskian compression.}
\label{tab:full}
\medskip
\begin{tabular}{@{}r r r r r@{}}
\toprule
$P_n$ & $g$ & $\log^2 P_n$ & $\rho$ & $q_W$ \\
\midrule
89        & 8   & 20.15   & 0.397  & 1.4912 \\
113       & 14  & 22.35   & 0.626  & 1.6408 \\
523       & 18  & 39.18   & 0.459  & 1.5461 \\
887       & 20  & 46.07   & 0.434  & 1.4210 \\
1\,129    & 22  & 49.41   & 0.445  & 1.3757 \\
1\,327    & 34  & 51.71   & 0.658  & 1.5407 \\
9\,551    & 36  & 83.99   & 0.429  & 1.3706 \\
15\,683   & 44  & 93.32   & 0.472  & 1.3817 \\
19\,609   & 52  & 97.69   & 0.532  & 1.3743 \\
31\,397   & 72  & 107.21  & 0.672  & 1.4240 \\
155\,921  & 86  & 142.97  & 0.602  & 1.3769 \\
360\,653  & 96  & 163.73  & 0.586  & 1.3457 \\
370\,261  & 112 & 164.40  & 0.681  & 1.3705 \\
492\,113  & 114 & 171.78  & 0.664  & 1.3428 \\
1\,349\,533 & 118 & 199.24 & 0.592 & 1.3331 \\
1\,357\,201 & 132 & 199.40 & 0.662 & 1.3364 \\
2\,010\,733 & 148 & 210.66 & 0.703 & 1.3385 \\
4\,652\,353 & 154 & 235.71 & 0.653 & 1.3039 \\
\bottomrule
\end{tabular}
\end{table}

\subsection{Key observations}

\begin{observation}[$q_W$ stabilization]\label{obs:stable}
For all maximal gaps with $g \ge 36$
(12 out of 18 data points), the Wronskian compression
lies in the band
\[
    q_W \;\in\; [1.3039, \; 1.4240].
\]
The range narrows further for $g \ge 86$: $q_W \in [1.30, 1.38]$.
Despite the gap length $g$ increasing by a factor of $4$
(from $36$ to $154$), $q_W$ exhibits no upward trend.
\end{observation}

\begin{observation}[Initial decay]\label{obs:decay}
For small gaps ($g < 36$), $q_W$ fluctuates between
$1.38$ and $1.64$, with the maximum $q_W = 1.6408$
occurring at $P_n = 113$, $g = 14$.
The early high values are driven by the dominance of
small primes ($2, 3, 5$) in short sequences, where factor
sharing is proportionally large relative to the number of
distinct prime factors introduced.
\end{observation}

\begin{observation}[Cramér ratio stability]\label{obs:cramer}
The Cram\'er ratio $\rho = g / \log^2 P_n$ fluctuates in
$[0.40, 0.70]$ with no clear trend toward $1$.
This is consistent with the known expectation that the
$\limsup$ is approached very slowly
(cf.~Granville's correction \cite{Granville1995},
which predicts $\limsup \rho \ge 2e^{-\gamma} \approx 1.123$).
\end{observation}

% ═══════════════════════════════════════════════════════════════════════
\section{Why Does $q_W$ Stabilize?}\label{sec:why}
% ═══════════════════════════════════════════════════════════════════════

The stabilization of $q_W$ is not obvious.
A naive expectation might be that longer composite
sequences share more prime factors (hence higher $q_W$),
but the data shows the opposite.
We identify two competing effects:

\subsection{The compression force: factor sharing}

In a sequence of $g - 1$ consecutive integers near $X$,
the number divisible by a prime $q \le g$ is
$\lfloor (g-1)/q \rfloor \ge 1$.
Each such prime contributes $\lfloor (g-1)/q \rfloor \cdot \log q$
to the numerator but only $\log q$ to the denominator.
By Mertens' theorem, the total ``shared'' contribution
from primes $q \le g$ is:
\[
    \text{(numerator excess)}
    \;\approx\;
    \sum_{q \le g} \left(\frac{g-1}{q} - 1\right) \log q
    \;\approx\;
    (g-1) \log g - g
    \;\sim\;
    g \log g.
\]

\subsection{The decompression force: large prime factors}

Each composite $c \in \mathcal{C}$ has at least one prime
factor $q > \sqrt{c} > \sqrt{X - g}$.
These large primes are almost certainly \emph{unique} to
their composite (they divide no other element of $\mathcal{C}$).
The number of such unique large primes grows roughly as
$\alpha \cdot (g - 1)$ for some proportion $\alpha > 0$.
Each contributes $\log q \approx \tfrac{1}{2} \log X$ to the
denominator without any sharing penalty.

The total ``unique'' contribution to the denominator is:
\[
    \alpha (g-1) \cdot \tfrac{1}{2}\log X.
\]

\subsection{The equilibrium}

The compression ratio is approximately:
\[
    q_W
    \;\approx\;
    \frac{(g-1)\log X}
    {\alpha(g-1) \cdot \tfrac{1}{2}\log X
    + \text{(small prime contributions)}}
    \;\approx\;
    \frac{\log X}{\tfrac{\alpha}{2}\log X + \theta(g)}
    \;\approx\;
    \frac{1}{\alpha/2 + \theta(g)/\log X}.
\]
As $X \to \infty$ with $g \sim c \log^2 X$
(Cram\'er regime), $\theta(g)/\log X \sim g/\log X
\sim c \log X \to \infty$, which would suggest
$q_W \to 0$.
But $\theta(g)/\log X = O(\log X)$ grows
\emph{much} slower than $\alpha(g-1)\log X / 2$,
so the large-prime term dominates, and
$q_W \to 2/\alpha$.
The observed $q_W \approx 1.3$ is consistent with
$\alpha \approx 1.5$, meaning roughly $2/3$ of the
composites in a gap contribute a large unique prime factor.

This heuristic explains the stabilization but does not
constitute a proof.
Making it rigorous would require quantitative control
over the distribution of large prime factors of consecutive
composites, which is closely related to deep questions in
sieve theory.

% ═══════════════════════════════════════════════════════════════════════
\section{Open Problems}\label{sec:open}
% ═══════════════════════════════════════════════════════════════════════

\begin{question}[Boundedness of $q_W$]\label{q:bounded}
Is $q_W(p_n, g_n)$ bounded as $n \to \infty$
for maximal prime gaps?
Our data suggests $q_W \le 1.65$ unconditionally and
$q_W \le 1.43$ for $g \ge 36$, but we have no proof.
\end{question}

\begin{question}[Limiting value]\label{q:limit}
Does $\lim_{n \to \infty} q_W(p_n, g_n)$ exist for
maximal gaps?
The data is consistent with convergence to a value in
$[1.20, 1.35]$, but the sample is too small to distinguish
convergence from slow oscillation.
\end{question}

\begin{question}[Connection to Cramér]\label{q:cramer}
Does the boundedness of $q_W$ imply (or follow from)
Cram\'er's conjecture?
A proof that $q_W \le C$ for some absolute constant
would imply, via the relation between the numerator and
denominator, a bound of the form
$g = O((\log X)^{C'})$ for some $C'$ depending on $C$.
Conversely, Cram\'er's conjecture $g = O(\log^2 X)$
would constrain the growth of the numerator, but
controlling the denominator requires additional input
about the radical.
\end{question}

\begin{question}[Beyond record gaps]\label{q:non_record}
Does $q_W$ stabilize for \emph{all} large prime gaps
(not just maximal ones)?
One expects that non-record gaps are ``less compressed''
(lower $q_W$), but this has not been systematically tested.
\end{question}

\begin{question}[Connection to the Erd\H{o}s--Pomerance conjecture]
\label{q:EP}
The stabilization of $q_W$ reflects the fact that
$\log \rad(\prod C_i)$ grows proportionally to
$\log(\prod C_i)$.
Is this related to conjectures on the distribution of
the radical function, such as those of
Erd\H{o}s and Pomerance on $\rad(n)$?
\end{question}

% ═══════════════════════════════════════════════════════════════════════
\section{Discussion}\label{sec:discussion}
% ═══════════════════════════════════════════════════════════════════════

\subsection{What is proven vs.\ what is observed}

\emph{Proven:} $q_W \ge 1$ always
(Proposition~\ref{prop:lower}).

\emph{Observed:}
$q_W$ stabilizes in $[1.30, 1.43]$ for record gaps
$g \ge 36$ up to $P_n = 4{,}652{,}353$
(Table~\ref{tab:full}).

\emph{Conjectured:} $q_W$ remains bounded for all
maximal prime gaps (Question~\ref{q:bounded}).

\subsection{Relation to the ABC conjecture}

The ABC conjecture constrains individual triples
$(a, b, c)$ with $a + b = c$, bounding the quality
$q(a,b,c) = \log c / \log \rad(abc)$.
The Wronskian compression $q_W$ is a \emph{different}
quantity: it measures the collective compression of
a \emph{sequence} of integers.
There is no known direct implication between the
ABC conjecture and the boundedness of $q_W$.

We emphasize this point to avoid a common confusion:
the ABC conjecture does not, in any known formulation,
bound the compression of a product of consecutive
composites.
Any future connection between $q_W$ and ABC would
require substantial new ideas.

\subsection{Relation to the algebraic structure papers}

In \cite{ChenSpin} and \cite{ChenOpp}, we proved that
Legendre intervals possess rigid quadratic residue
distributions (the Spin Asymmetry Theorem and the
Oppermann Parity Law).
Those results concern the \emph{additive/algebraic} axis
(residues modulo $p = 2n - 1$), while the present paper
concerns the \emph{multiplicative} axis (radical of
consecutive composites).
A unified framework connecting these two axes---the
algebraic structure of residues and the multiplicative
structure of factorizations---remains a distant goal.

\subsection*{Data and code availability}

The scanning script and complete dataset are available at
\url{https://github.com/Ruqing1963/cramer-wronskian-stability}.

% ═══════════════════════════════════════════════════════════════════════
\begin{thebibliography}{99}

\bibitem{ChenSpin}
  R.~Chen,
  ``Algebraic rigidity and quadratic residue asymmetry
  in Legendre intervals,''
  Zenodo, 2026.
  \url{https://zenodo.org/records/18706876}

\bibitem{ChenOpp}
  R.~Chen,
  ``Oppermann's parity law: quadratic residue symmetry
  breaking in half-intervals via the negation involution,''
  Zenodo, 2026.
  \url{https://zenodo.org/records/18707265}

\bibitem{ChenFF}
  R.~Chen,
  ``The geometry of prime vacuums: Legendre's conjecture
  in function fields via monodromy and Chebotarev density,''
  Zenodo, 2026.
  \url{https://zenodo.org/records/18705744}

\bibitem{Cramer1936}
  H.~Cram\'er,
  ``On the order of magnitude of the difference between
  consecutive prime numbers,''
  \textit{Acta Arith.}\ \textbf{2} (1936), 23--46.

\bibitem{Granville1995}
  A.~Granville,
  ``Harald Cram\'er and the distribution of prime numbers,''
  \textit{Scand.\ Actuar.\ J.}\ \textbf{1995} (1995), 12--28.

\end{thebibliography}

\end{document}
